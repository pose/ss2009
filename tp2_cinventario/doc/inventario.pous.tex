%% LyX 1.6.1 created this file.  For more info, see http://www.lyx.org/.
%% Do not edit unless you really know what you are doing.
\documentclass[spanish]{article}
\usepackage[T1]{fontenc}
\usepackage[utf8]{inputenc}

\usepackage{babel}
\addto\shorthandsspanish{\spanishdeactivate{~<>}}

\begin{document}
Como se detalla en \cite{diaz09} se plantea un sistema de control
de inventario sigue una dinámica que cumple las siguientes propiedades:
\begin{enumerate}
\item Los productos son homogéneos
\item La cantidad de productos manipulada en cada ejercicio es muy grande
\end{enumerate}
Recordando la notación introducida se define: $x_{1}$como el nivel
de inventario y $x_{2}$como la tasa de ventas del producto. Además,
se especifica la siguiente relación:

\begin{equation}
\frac{dx_{2}}{dt}(t)=-Ku(t)\label{eq:la_k}\end{equation}


donde K > 0.

A modo de extensión de esta propuesta se define a $\dot{x_{1}}$de
la siguiente manera:

\begin{equation}
\dot{x_{1}}=u(t)-x_{2}\label{eq:dotx1}\end{equation}


Donde $u(t)$ es la tasa de producción de la empresa. Esto sería equivalente
a decir que la variación de la cantidad de productos depende de lo
que se produce (entra) y de lo que se vende (sale):
\begin{enumerate}
\item Si $x_{2}<u(t)$ el nivel de inventario va a aumentar: $\dot{x_{1}}>0$.
Esto se debe a que se produce más de lo que se vende.
\item Mientras que cuando $x_{2}>u(t)$ el inventario disminuye más rápido
que se repone: $\dot{x_{1}}<0$. Esto es causado a que se vende más
de lo que se produce.
\item Cuando $x_{2}=u(t)$, por lo tanto$\dot{x_{1}}=0$, la cantidad de
elementos en el inventario permanece igual. Esto puede atribuirse
a que se vende tanto como lo que se produce o a que no se vende ni
se produce.
\end{enumerate}

\section{Controlando el inventario}

Se desea controlar el nivel del inventario tomando como output $y(t)=x_{1}(t)$
siendo $r(t)$ el setpoint de referencia. Esto es decir:

\begin{equation}
u(t)=r(t)-y(t)\end{equation}


Rango de valores K donde el sistema es asintóticamente estable

\bibliographystyle{plain}
\nocite{*}
\bibliography{references}

\end{document}
