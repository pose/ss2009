\documentclass{sig-alternate}
%\documentclass[a4paper,10pt]{article}
\usepackage[utf8]{inputenc}
%El template pincha con spanish babel
%\usepackage[spanish]{babel}
\usepackage{amsmath, amssymb, amsfonts}
\usepackage{soul}
\usepackage{graphicx}
\usepackage{fancybox}
\usepackage[table]{xcolor}
\usepackage{soul}

\title{Banzai Totsugeki} 

\numberofauthors{4}
\author{
\alignauthor
Pose, Alberto Miguel\\
       \affaddr{Instituto Tecnol\'ogico de Buenos Aires}\\
       \affaddr{Buenos Aires, Argentina}\\
       \email{apose@alu.itba.edu.ar}
\alignauthor
Catalano, Juan Ignacio\\
       \affaddr{Instituto Tecnol\'ogico de Buenos Aires}\\
       \affaddr{Buenos Aires, Argentina}\\
       \email{jcatalan@alu.itba.edu.ar}
\and
\alignauthor 
Palombo, Mart\'i�n\\
       \affaddr{Instituto Tecnol\'ogico de Buenos Aires}\\
       \affaddr{Buenos Aires, Argentina}\\
       \email{mpalombo@alu.itba.edu.ar}
\alignauthor 
V\'azquez, Santiago Jos\'e\\
       \affaddr{Instituto Tecnol\'ogico de Buenos Aires}\\
       \affaddr{Buenos Aires, Argentina}\\
       \email{savazque@alu.itba.edu.ar}
}

\date{}

\begin{document}

\maketitle

\begin{abstract}
ABASTRAEEME
\end{abstract} 

\newpage

\section{Introducci\'on}

PONER QUE TIENE QUE CUMPLIR CADA METODO indep uniform
\section{Atacando a L'Ecuyer}
\label{sec:goingdown}

Se realiza un an\'alisis de PONER EL TIPO DE ANALISIS PARA CADA METODO.
Se hace una realizaci\'on de $10000$ muestras del generador de L'Ecuyer y se realian dichos tests.
Cabe destacar que para la realizaci\'on del generador de L'Ecuyer se utilizan las semillas SEMILLAME y 
se agrupa dicha realizaci\'on en $10$ intervalos de clase.

\subsection{Test Chi Cuadrado}
\label{sec:chi}
Se aplica el test de Chi Cuadrado y se obtiene el estad\'istico $\chi_{0}^{2}$=RESULTADO.
Entonces, para $9$ grados de libertad y una significaci\'on de $\alpha=0.05$ se obtiene
el valor cr\'itico $\chi_{9,0.05}^{2}=16.92$.
Como $\chi_{0}^{2}=RESULTADO \geq \chi_{9,0.05}^{2}=16.92$ se REACHAZA/ACEPTA la hip\'otesis $H_{0}$
de que la muestra provenga de una distribuci\'on uniforme.

\subsection{Test Kolmogorov-Smirnov}
\label{sec:kolmogorov}
Se aplica el test de Kolmogorov-Smirnov y se obtiene $D=VALOR$.
Para un nivel de significaci\'on $\alpha=0.05$ y $10$ intervalos de clase
se obtiene $D_{0.05}=0.410$. Entonces como $D MAYOR/MENOR D_{\alpha}$
se ACEPTA/RECHAZA la hip\'otesis $H_{0}$
de que la muestra provenga de una distribuci\'on uniforme.


\section{Transformando a L'Ecuyer}
\label{sec:triangle}

A partir del generador de L'Ecuyer se desea obtener un generador de n\'umeros
pseudoaleatorios distribuidos seg\'un una funci\'on densidad triangular.\\
EXPLICAR EL ALGORITMO HAY UN TEOREMA LOCO
La funci\'on de densidad triangular es \eqref{eq:triangle}.

\begin{equation}
\label{eq:triangle}
f_{X}(x) =
\begin{cases}
\frac{2(x-a)}{(b-a)(c-a)} \quad & \text{si } a\leq x \leq b \\
\frac{2(c-x)}{(c-b)(c-a)} \quad & \text{si } b < x \leq c \\
0 \quad & \text{en otro caso}
\end{cases}
\end{equation}

Hallamos la funci\'on inversa \eqref{eq:inverse}.\\

\begin{equation}
\label{eq:triangle}

CAMBIAR POR LA INVERSA

f_{X}(x) =
\begin{cases}
\frac{2(x-a)}{(b-a)(c-a)} \quad & \text{si } a\leq x \leq b \\
\frac{2(c-x)}{(c-b)(c-a)} \quad & \text{si } b < x \leq c \\
0 \quad & \text{en otro caso}
\end{cases}
\end{equation}


Luego, el algoritmo consiste en COMPLETAR


\newpage

\section{Resultados Conclusiones}
\label{sec:conclusiones}

%PONER QUE CHI SALE DE http://www.wiphala.net/research/manual/statistic/chi_cuadrado.html
%PONER QUE LOS DATOS DE KOLMOGOROV SE SACAN DE http://www.eridlc.com/onlinetextbook/appendix/table7.htm

\end{document}
