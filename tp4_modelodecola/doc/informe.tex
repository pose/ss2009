\documentclass{sig-alternate}
%\documentclass[a4paper,10pt]{article}
\usepackage[utf8]{inputenc}
%El template pincha con spanish babel
%\usepackage[spanish]{babel}
\usepackage{amsmath, amssymb, amsfonts}
\usepackage{graphicx}
\usepackage{fancybox}
\usepackage[table]{xcolor}
\usepackage{soul}

\newtheorem{theorem}{Teorema}

\title{Que bien que est\'a esa cola} 

\numberofauthors{4}
\author{
\alignauthor
Pose, Alberto Miguel\\
       \affaddr{Instituto Tecnol\'ogico de Buenos Aires}\\
       \affaddr{Buenos Aires, Argentina}\\
       \email{apose@alu.itba.edu.ar}
\alignauthor
Catalano, Juan Ignacio\\
       \affaddr{Instituto Tecnol\'ogico de Buenos Aires}\\
       \affaddr{Buenos Aires, Argentina}\\
       \email{jcatalan@alu.itba.edu.ar}
\and
\alignauthor 
Palombo, Mart\'in\\
       \affaddr{Instituto Tecnol\'ogico de Buenos Aires}\\
       \affaddr{Buenos Aires, Argentina}\\
       \email{mpalombo@alu.itba.edu.ar}
\alignauthor 
V\'azquez, Santiago Jos\'e\\
       \affaddr{Instituto Tecnol\'ogico de Buenos Aires}\\
       \affaddr{Buenos Aires, Argentina}\\
       \email{savazque@alu.itba.edu.ar}
}

\date{}

\begin{document}

\maketitle

\begin{abstract}
Se estudia el modelo de cola M/M/1. ABSTRAEME
\end{abstract} 

\newpage

\section{Introducci\'on}

Se analiza el modelo de Cola simple mediante la simulaci\'on por eventos discretos.
La estructura din\'amica de dicha cola se la indica como M/M/1/$\infty$/FIFO, o simplemente M/M/1.
En este modelo, se asume que los clientes llegan al sistema mediante un proceso de Poisson con una tasa
media de $\lambda$ [clientes/hora] y el servidor atiende a cada cliente con tiempo de servicio exponencialmente
distribuido con media 1/$\mu$. \\
Si el servidor est\'a desocupado y llega un cliente, entonces este se sirve inmediatamente. Si el servidor est\'a
ocupado, entonces el cliente que llega entra en cola.

PRESENTAR SECCIONES

\section{Atacando a L'Ecuyer}
\label{sec:goingdown}

\newpage

\section{Resultados y Conclusiones}
\label{sec:conclusiones}
CONCLUIME

% TODO ver si corresponde
% \begin{thebibliography}{10}
% \bibitem{chiTable} Tabla $\chi^{2}$.
% \begin{verbatim}
% http://www.wiphala.net/research/manual/
% statistic/chi_cuadrado.html
% \end{verbatim}
% \bibitem{KSTable} Tabla Kolmogorov-Smirnov.
% \begin{verbatim}
% http://www.eridlc.com/onlinetextbook/
% appendix/table7.htm
% \end{verbatim} 
% \bibitem{TStudentTable} Tabla T Student. \begin{verbatim}www.elosiodelosantos.com/sergiman/archivos/tablat.xls\end{verbatim}
% \end{thebibliography}
\end{document}